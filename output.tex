\documentclass{article}
\usepackage[margin=1in]{geometry}
\usepackage[utf8]{inputenc}
\usepackage[]{graphicx}
\usepackage[]{amsmath}
\usepackage[]{tabularx}
\usepackage[]{listings}
\usepackage[]{xcolor}
\usepackage[]{wrapfig}
\usepackage[]{seqsplit}
\usepackage[]{soul}
\usepackage[backend=biber,style=authoryear]{biblatex}
\title{Untitled}\author{Anonymous}\date{\today}
\setlength{\parindent}{0pt}
\lstdefinestyle{defaultstyle}{backgroundcolor=\color{white},basicstyle=\ttfamily\footnotesize,breakatwhitespace=false,breaklines=true,captionpos=b,commentstyle=\color[rgb]{0.1,0.5,0.1},keywordstyle=\color[rgb]{0.1,0.1,0.7},stringstyle=\color[rgb]{0.7,0.1,0.1},identifierstyle=\color{black},numberstyle=\tiny\color[rgb]{0.5,0.5,0.5},stepnumber=1,numbersep=10pt,showspaces=false,showstringspaces=false,showtabs=false,tabsize=4,frame=single,rulecolor=\color[rgb]{0.5,0.5,0.5},title=\lstname}

\begin{document}
\section*{The basics:}\begin{quote}

Welcome to Texpile, a \textbf{WYSIWYM} \seqsplit{(What-You-See-Is-What-You-Mean)} editor. Texpile uses a compiler to convert your document into a specific format based on a chosen template. To see the transformed document, click the "Compile" button in the top right corner. Your final document will be displayed in the Preview Box on the right.

\end{quote}


Texpile will also also highlight some of you're mistake in the document. Click on the highlighted texts to fix these possble errors.


\section*{But there is more!}\subsection*{Mathematics}
Texpile comes within \ul{first party} mathmatics editing support.

You can use the shortcut \texttt{CTRL-M or CMD-M} to create an inline math field to type your equation. Like this one: $\lim_{x\to3}3x+1$

You can also use the shortcut \texttt{CTRL-SHIFT-M or CMD-SHIFT-M} to create a block math field Like this one:


\begin{equation}
C = A \times B = \begin{bmatrix}
a_{11} & a_{12} \\
a_{21} & a_{22}
\end{bmatrix}
\begin{bmatrix}
b_{11} & b_{12} \\
b_{21} & b_{22}
\end{bmatrix}
= \begin{bmatrix}
a_{11}b_{11} + a_{12}b_{21} & a_{11}b_{12} + a_{12}b_{22} \\
a_{21}b_{11} + a_{22}b_{21} & a_{21}b_{12} + a_{22}b_{22}
\end{bmatrix}
\end{equation}\subsection*{Code blocks}
Texpile allows you to create code blocks. Press \texttt{CTRL-SHIFT-`:}

\begin{lstlisting}[style=defaultstyle, language=Java, caption={}]
class Main{
  public static void main(String args[]){
    System.out.println("Hi");
  }
}
\end{lstlisting}
\subsection*{Tables}
\begin{tabularx}{\textwidth}{|X|X|}
\hline

Texpile creates \seqsplit{professional-looking} documents with ease.
 &
Texpile is versatile: you can use one of the prebuilt templates or build your own.
 \\\hline
Texpile saves your document online so you can access it anywhere on any device.
 &
Texpile uses LaTeX to produce your final document, but you will never have to worry about compile errors again.
 \\\hline
\end{tabularx}
\subsection*{Images}
You can add images by simply pasting or dragging it here \cite{doe2021conference}.



\begin{figure}[h]
\centering
\includegraphics[width=0.5857142857142857\textwidth]{x_square.png}
\caption{Figure 1: Image}
\end{figure}



\printbibliography
\end{document}