\documentclass[12pt,letterpaper]{article}
\usepackage[american]{babel}
\usepackage{csquotes}
\usepackage[style=mla,backend=biber]{biblatex}
\usepackage[letterpaper,top=1.0in,bottom=1.0in,left=1.0in,right=1.0in]{geometry}
\usepackage{times}
\usepackage{setspace}
\usepackage{rotating}
\usepackage{fancyhdr}
\usepackage[utf8]{inputenc}
\usepackage{tabularx}
\usepackage{amsmath}
\usepackage{graphicx}
\usepackage{xcolor}
\doublespacing
\pagestyle{fancy}
\lhead{}
\chead{}
\rhead{ \thepage}
\lfoot{}
\cfoot{}
\rfoot{}
\renewcommand{\headrulewidth}{0pt}
\renewcommand{\footrulewidth}{0pt}
\setlength\headsep{0.333in}
\newcommand{\bibent}{\noindent\hangindent 40pt}
\newenvironment{workscited}{\newpage\begin{center}Works Cited\end{center}}{\newpage}
\renewcommand{\maketitle}{\makemlaheader}
\newcommand{\makemlaheader}{ \\  \\  \\ \today\\\begin{center}\textnormal{Untitled}\end{center}}
\addbibresource{references.bib}
\setlength{\parindent}{0.5in}

\begin{document}
\noindent Anonymous\\Unknown\\Unknown\\\today\\
\begin{center}\textnormal{Untitled}\end{center}
\begin{flushleft}
\section{The basics:}
\begin{quote}

Welcome to Texpile, a WYSIWYM (What-You-See-Is-What-You-Mean) editor. Texpile uses a compiler to convert your document into a specific format based on a chosen template. To see the transformed document, click the "Compile" button in the top right corner. Your final document will be displayed in the Preview Box on the right.

\end{quote}


Texpile will also also highlight some of you're mistake in the document. Click on the highlighted texts to fix these possble errors.


\section{But there is more!}
\subsection{Mathematics}

Texpile comes within first party mathmatics editing support.

You can use the shortcut CTRL-M or CMD-M to create an inline math field to type your equation. Like this one: $\lim_{x\to3}3x+1$

You can also use the shortcut CTRL-SHIFT-M or CMD-SHIFT-M to create a block math field Like this one:


\[
C = A \times B = \begin{bmatrix}
a_{11} & a_{12} \\
a_{21} & a_{22}
\end{bmatrix}
\begin{bmatrix}
b_{11} & b_{12} \\
b_{21} & b_{22}
\end{bmatrix}
= \begin{bmatrix}
a_{11}b_{11} + a_{12}b_{21} & a_{11}b_{12} + a_{12}b_{22} \\
a_{21}b_{11} + a_{22}b_{21} & a_{21}b_{12} + a_{22}b_{22}
\end{bmatrix}
\]\subsection{Code blocks}

Texpile allows you to create code blocks. Press CTRL-SHIFT-`:

\begin{verbatim}
class Main{
  public static void main(String args[]){
    System.out.println("Hi");
  }
}
\end{verbatim}
\subsection{Tables}

\begin{tabularx}{\textwidth}{|X|X|}
\hline

Texpile creates professional-looking documents with ease.
 &
Texpile is versatile: you can use one of the prebuilt templates or build your own.
 \\\hline
Texpile saves your document online so you can access it anywhere on any device.
 &
Texpile uses LaTeX to produce your final document, but you will never have to worry about compile errors again.
 \\\hline
\end{tabularx}
\subsection{Images}

You can add images by simply pasting or dragging it here \cite{doe2021conference}.



\begin{figure}[h]
\centering
\includegraphics[width=0.5857142857142857\textwidth]{x_square.png}
\caption{Figure 1: Image}
\end{figure}



\end{flushleft}
\begin{workscited}\printbibliography[heading=none]\end{workscited}
\end{document}